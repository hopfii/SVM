\documentclass[a4paper,11pt,twoside]{scrreprt}

\usepackage[utf8]{inputenc}
\usepackage[T1]{fontenc}   
\usepackage{graphicx}       
\usepackage[german]{babel}
\usepackage{csquotes}     
\usepackage{acronym}
\usepackage{eurosym}
\usepackage[linktocpage=true]{hyperref}
\usepackage[bindingoffset=8mm]{geometry}
\usepackage{caption}
\captionsetup{format=hang, justification=raggedright}
\usepackage[style=authoryear, backend=biber]{biblatex}
\usepackage{float}
\usepackage{rotating}
\usepackage{blkarray}
\usepackage{amsmath}
\usepackage{amssymb}
\usepackage{gensymb}
\usepackage{amsthm}
\newtheorem{theorem}{Theorem}[section]
\newtheorem{lemma}[theorem]{Lemma}
\usepackage{listings}
\addbibresource{references.bib} 
\usepackage{caption}
\usepackage{subcaption}

\newcommand{\argmin}[1]{\underset{#1}{\operatorname{arg}\,\operatorname{min}}\;}

\begin{document}

% Titelblatt:
% \newpage\mbox{}\newpage
\cleardoublepage   % force output to a right page
\thispagestyle{empty}
\begin{titlepage}
  \begin{flushright}
  \includegraphics[width=0.4\linewidth]{assets/Logo-A3.jpg}
  \end{flushright}
  \begin{center}
  \section*{Support Vector Machines (SVM)}
  \vspace{2cm}

\textbf{Computational Intelligence II}    
\vspace{0.5cm}

  Informatik - Software and Information Engineering\\
  Fachhochschule Vorarlberg\\

  \vspace{1cm}
  
    Erstellt von\\
  André Hopfgartner \& Matthias Rupp\\
  
 
  \vspace{1cm}
  

  
  Dornbirn, am \today
  
  
  \end{center}
\end{titlepage}


% Inhaltsverzeichnis:
\clearpage   % force output to a right page
\setcounter{tocdepth}{2}
\setcounter{secnumdepth}{4}
\tableofcontents

% evtl. Abkürzungsverzeichnis:
\clearpage
\phantomsection
\addcontentsline{toc}{chapter}{Abkürzungsverzeichnis}
\chapter*{Abkürzungsverzeichnis}
\begin{acronym}
 \acro{SVM}{Support Vector Machine}
\end{acronym}



\chapter{Einführung}

\section{Intuition}
Ziel: möglichst breites Band zwischen den 2 verschiedenen Klassen aufziehen.

\section{Mathematische Herleitung}
Gegeben sei ein Gewichtsvektor $w \in \mathbb{R}^{D}$, ein Bias $b \in \mathbb{R}$ und ein beliebiger Punkt $x \in \mathbb{R}^{D}$. Eine Ebene im Raum kann definiert werden durch:

\begin{equation} \label{plane_eq}
    \begin{aligned}
    w^{T} x + b &= 0 \\
    \end{aligned}
\end{equation}

Weil \autoref{plane_eq} mit verschiedenen Skalarwerten skaliert werden kann, führen wir eine zusätzliche Bedingung ohne Beschränkung der Allgemeinheit ein. Sei $x_{n} \in \mathbb{R}^{D}$ der am nächsten zur Ebene gelegene Punkt so soll gelten:

\begin{equation} \label{plane_normalization}
	\begin{aligned}
		|w^{T} x_{n} + b| &= 1 \\
	\end{aligned}
\end{equation}

Als nächsten Schritt bestimmen wir den euklidischen Normalabstand $D$ eines beliebigen Punkts $x_{k} \in \mathbb{R}^{D}$ zu der Ebene. Hierfür ist zuerst zu bemerken, dass $w$ normal zur definierten Ebene steht.

\begin{lemma}
	Eine Ebene sei definiert durch $w^{T} x + b = 0$. Der Vektor $w$ steht normal zu der definierten Ebene.
\end{lemma}

\begin{proof}
	Man wähle zwei Punkte $x_{1}, x_{2} \in \mathbb{R}^{D}$ die auf der Ebene liegen. Somit muss gelten:
	\begin{equation}
		\begin{aligned}
			w^{T} x_{1} + b &= 0 \\
			w^{T} x_{2} + b &= 0 \\
			w^{T} (x_{1} - x_{2}) &= 0 \leftrightarrow \lVert w^{T} \rVert \lVert x_{1} - x_{2} \rVert \cos(\alpha) = 0 \leftrightarrow \alpha = 90^{\circ}
		\end{aligned}
	\end{equation}
\end{proof}

Um den Normalabstand $D$ eines beliebigen Punkts $x_{k}$ zu ermitteln wählt man einen Punkt $x$ der auf der Ebene liegt und projiziert den Vektor $(x_{k} - x)$ auf den Einheitsvektor von $w$. Weil nur der tatsächliche Abstand zur Ebene relevant ist nimmt man den Betrag.

\begin{equation} \label{distance_to_plane}
	\begin{aligned}
		D &= | \frac{w^{T}}{\lVert w \rVert} (x_{k} - x) | = \\
		&= \frac{1}{\lVert w \rVert} | (w^{T} x_{k} - w^{T} x) | =\\
		&= \frac{1}{\lVert w \rVert} | (w^{T} x_{k} + b - (w^{T} x + b)) |
	\end{aligned}
\end{equation}

Weil der Punkt $x$ auf der Ebene liegt gilt $w^{T} x + b = 0$ (\autoref{plane_eq}):
\begin{equation} \label{distance_to_plane_simplified1}
	\begin{aligned}
		D &= \frac{1}{\lVert w \rVert} | (w^{T} x_{k} + b) |
	\end{aligned}
\end{equation}

Aus \autoref{plane_normalization} gilt weiters $| (w^{T} x_{k} + b) | = 1$:
\begin{equation} \label{distance_to_plane_simplified2}
	\begin{aligned}
		D &= \frac{1}{\lVert w \rVert}
	\end{aligned}
\end{equation}



% Literaturverzeichnis:
\clearpage
\phantomsection
\addcontentsline{toc}{chapter}{Bibliography}
\printbibliography

\end{document}
