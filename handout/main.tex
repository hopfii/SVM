\documentclass[a4paper,11pt,twoside]{scrreprt}

\usepackage[utf8]{inputenc}
\usepackage[T1]{fontenc}   
\usepackage{graphicx}       
\usepackage[german]{babel}
\usepackage{csquotes}     
\usepackage{acronym}
\usepackage{eurosym}
\usepackage[linktocpage=true]{hyperref}
\usepackage[bindingoffset=8mm]{geometry}
\usepackage{caption}
\captionsetup{format=hang, justification=raggedright}
\usepackage[style=authoryear, backend=biber]{biblatex}
\usepackage{float}
\usepackage{rotating}
\usepackage{blkarray}
\usepackage{amsmath}
\usepackage{amssymb}
\usepackage{gensymb}
\usepackage{listings}
\addbibresource{references.bib} 
\usepackage{caption}
\usepackage{subcaption}

\newcommand{\argmin}[1]{\underset{#1}{\operatorname{arg}\,\operatorname{min}}\;}

\begin{document}

% Titelblatt:
% \newpage\mbox{}\newpage
\cleardoublepage   % force output to a right page
\thispagestyle{empty}
\begin{titlepage}
  \begin{flushright}
  \includegraphics[width=0.4\linewidth]{assets/Logo-A3.jpg}
  \end{flushright}
  \begin{center}
  \section*{Support Vector Machines (SVM)}
  \vspace{2cm}

\textbf{Computational Intelligence II}    
\vspace{0.5cm}

  Informatik - Software and Information Engineering\\
  Fachhochschule Vorarlberg\\

  \vspace{1cm}
  
    Erstellt von\\
  André Hopfgartner \& Matthias Rupp\\
  
 
  \vspace{1cm}
  

  
  Dornbirn, am \today
  
  
  \end{center}
\end{titlepage}


% Inhaltsverzeichnis:
\clearpage   % force output to a right page
\setcounter{tocdepth}{2}
\setcounter{secnumdepth}{4}
\tableofcontents

% evtl. Abkürzungsverzeichnis:
\clearpage
\phantomsection
\addcontentsline{toc}{chapter}{Abkürzungsverzeichnis}
\chapter*{Abkürzungsverzeichnis}
\begin{acronym}
 \acro{SVM}{Support Vector Machine}
\end{acronym}



\chapter{Einführung}

\section{Intuition}
Ziel: möglichst breites Band zwischen den 2 verschiedenen Klassen aufziehen.

\section{Mathematische Herleitung}
Gegeben sei ein Gewichtsvektor $\vec{w} \in \mathbb{R}^{M}$ und ein beliebiger Vektor $\vec{u} \in \mathbb{R}^{M}$. Der Gewichtsvektor $\vec{w}$ spannt mit dem Bias $b \in \mathbb{R}$ eine Ebene im Raum auf. Um entscheiden zu können, ob ein Punkt $\vec{u}$ über oder unter der Trennebene liegt, kann folgende Entscheidungsregel eingeführt werden:

\begin{equation} \label{dec_rule}
    \begin{aligned}
    \vec{w} \cdot \vec{u} + b & \geq 0 \\
    \end{aligned}
\end{equation}

Als nächstes führen wir ein Band um die Trennebene ein. Gegeben sei eine Trainingsmenge von Tupeln $(\vec{x_{i}}, d_{i})$ mit $\vec{x_{i}} \in \mathbb{R}^{M}$ und $d_{i} \in \{-1, +1\}$. Bei $\vec{x_{i}}$ handelt es sich um einen Eingabevektor, $d_{i}$ ist das jeweils zugehörige Label. Wir führen folgende Regeln ein:
\begin{equation} \label{band_rules}
    \begin{aligned}
    \vec{w} \cdot \vec{x_{i}} + b & \geq +1 \textbf{ für $d_{i} = +1$} \\
    \vec{w} \cdot \vec{x_{i}} + b & \leq -1 \textbf{ für $d_{i} = -1$}
    \end{aligned}
\end{equation}

\autoref{band_rules} kann weiter verallgemeinert werden, indem auf beiden Seiten mit $d_{i}$ multipliziert wird. Weil für jedes beliebige $d_{i}$ gilt $d_{i} d_{i} = 1$ ergibt sich:

\begin{equation} \label{band_rule}
    \begin{aligned}
    d_{i} (\vec{w} \cdot \vec{x_{i}} + b) & \geq 1 \textbf{ für $d_{i} = +1$}\\
    d_{i} (\vec{w} \cdot \vec{x_{i}} + b) & \geq 1 \textbf{ für $d_{i} = -1$}\\
    \end{aligned}
\end{equation}

Für Punkte $\vec{x_{i}}$, die genau auf den Grenzen des Bandes liegen, muss also gelten:
\begin{equation} \label{band_rule}
    \begin{aligned}
    d_{i} (\vec{w} \cdot \vec{x_{i}} + b) -1 &= 0\\
    \end{aligned}
\end{equation}

Die Breite $\xi$ des Bandes lässt sich berechnen, indem zwei Punkte $\vec{x_{+}}$ und $\vec{x_{-}}$ gewählt werden, die jeweils auf den Grenzen des Bandes liegen. Der Differenzvektor $\vec{x_{+}}$ - $\vec{x_{-}}$ kann anschließend auf den Einheitsvektor $\vec{w}^{0}$ projiziert werden mittels des Skalarprodukts. Als Ergebnis erhält man den Normalabstand zwischen den Grenzen des Bandes, was genau der Breite des Bandes entspricht:

\begin{equation} \label{band_width}
    \begin{aligned}
    \xi &= (\vec{x_{+}} - \vec{x_{-}}) \cdot \frac{\vec{w}}{\lVert \vec{w} \rVert}\\
    &=  \frac{1}{\lVert \vec{w} \rVert} \cdot (\vec{x_{+}} \cdot \vec{w} - \vec{x_{-}} \cdot \vec{w})
    \end{aligned}
\end{equation}

Setzt man in \autoref{band_rule} die Tupel $(\vec{x_{+}}, +1)$ und $(\vec{x_{-}}, -1)$ ein erhält man:
\begin{equation} \label{band_width_x}
    \begin{aligned}
     \vec{w} \cdot \vec{x_{+}} &= 1 -b \textbf{ für $d_{i} = +1$}\\
     \vec{w} \cdot \vec{x_{-}} &= -b -1 \textbf{ für $d_{i} = -1$}\\
    \end{aligned}
\end{equation}

Damit ergibt sich für \autoref{band_width}:
\begin{equation} \label{band_width_final}
    \begin{aligned}
    \xi &= \frac{1}{\lVert \vec{w} \rVert} \cdot ((1-b) - (-b-1)) \\
    &= \frac{2}{\lVert \vec{w} \rVert}
    \end{aligned}
\end{equation}

Weil wir bei \ac{SVM} ein möglichst breites Band zwischen den Trainingsdaten haben möchten führen wir eine Maximierung der Breite durch, die sich auch als Minimierung darstellen lässt:
\begin{equation} \label{band_width_max}
    \begin{aligned}
    \hat{\xi} = \max \frac{2}{\lVert \vec{w} \rVert} &= \min {\lVert \vec{w} \rVert} &= \frac{1}{2} \min {\lVert \vec{w} \rVert}^{2}
    \end{aligned}
\end{equation}

Die in \autoref{band_width_max} beschriebene Umformungen wurden gemacht, weil dadurch die später zu optimierende Gleichung besser darstellbar ist. \\

Um \autoref{band_width_max} optimieren zu können wird die Lagrangegleichung mit der in \autoref{band_rule} beschriebenen Nebenbedingung aufgestellt:
\begin{equation} \label{lagrange_1}
    \begin{aligned}
    L &= \frac{1}{2} {\lVert \vec{w} \rVert}^{2} - \sum (\alpha_{i} (d_{i} (\vec{w} \cdot \vec{x_{i}} + b) -1)) 
    \end{aligned}
\end{equation}

Um den Extremwert einer Lagrangegleichung ermitteln zu können müssen alle partiellen Ableitungen 0 gesetzt werden:

\begin{equation} \label{lagrange_partial_w}
    \begin{aligned}
    \frac{\partial L}{\partial \vec{w}} = \vec{w} - \sum (\alpha_{i} d_{i} \vec{x_{i}}) &\overset{!}{=} 0 \\
    \vec{w} &= \sum (\alpha_{i} d_{i} \vec{x_{i}})
    \end{aligned}
\end{equation}
\newline
\begin{equation} \label{lagrange_partial_b}
    \begin{aligned}
    \frac{\partial L}{\partial b} = - \sum (\alpha_{i} d_{i}) &\overset{!}{=} 0 \\
    \sum (\alpha_{i} d_{i}) &= 0
    \end{aligned}
\end{equation}

Die Ergebnisse von \autoref{lagrange_partial_w} und \autoref{lagrange_partial_b} können in \autoref{lagrange_1} wieder eingesetzt werden:
\begin{equation} \label{lagrange_2}
    \begin{aligned}
    L &= \frac{1}{2} (\sum (\alpha_{i} d_{i} \vec{x_{i}})) \cdot (\sum (\alpha_{j} d_{j} \vec{x_{j}})) - \sum (\alpha_{i} d_{i} x_{i} \cdot (\sum (\alpha_{j} d_{j} \vec{x_{j}})) - \sum (\alpha_{i} d_{i} b) + \sum (\alpha_{i}) =\\
    &= \frac{1}{2} (\sum (\alpha_{i} d_{i} \vec{x_{i}})) \cdot (\sum (\alpha_{j} d_{j} \vec{x_{j}})) - \sum (\alpha_{i} d_{i} x_{i} \cdot (\sum (\alpha_{j} d_{j} \vec{x_{j}})) - b \sum (\alpha_{i} d_{i}) + \sum (\alpha_{i})
    \end{aligned}
\end{equation}

Aus \autoref{lagrange_partial_b} gilt $\sum (\alpha_{i} d_{i}) = 0$:
\begin{equation} \label{lagrange_3}
    \begin{aligned}
    L &= \frac{1}{2} (\sum (\alpha_{i} d_{i} \vec{x_{i}})) \cdot (\sum (\alpha_{j} d_{j} \vec{x_{j}})) - \sum (\alpha_{i} d_{i} x_{i} \cdot (\sum (\alpha_{j} d_{j} \vec{x_{j}})) + \sum (\alpha_{i})
    \end{aligned}
\end{equation}

Bei den Summen ergibt sich eine Symmetrie der Form $\frac{1}{2} a - a$ wobei $a$ als Abkürzung für die Summen verstanden werden kann:
\begin{equation} \label{lagrange_final}
    \begin{aligned}
    L &= \frac{1}{2} (\sum \sum (\alpha_{i} \alpha_{j} d_{i} d_{j} \vec{x_{i}} \cdot \vec{x_{j}})) + \sum (\alpha_{i})
    \end{aligned}
\end{equation}

Das bemerkenswerte an \autoref{lagrange_final} ist, dass die Optimierung nur von dem Skalarprodukt von je 2 Samples abhängt. \\

Weiters lässt sich die Entscheidungsregel \autoref{dec_rule} mit \autoref{lagrange_partial_w} umformen:
\begin{equation} \label{dec_rule}
    \begin{aligned}
    (\sum (\alpha_{i} d_{i} \vec{x_{i}})) \cdot \vec{u} + b & \geq 0 
    \end{aligned}
\end{equation}

Somit hängt auch die Entscheidungsregel nur von den Skalarprodukten ab.



% Literaturverzeichnis:
\clearpage
\phantomsection
\addcontentsline{toc}{chapter}{Bibliography}
\printbibliography

\end{document}
